\setcounter{chapter}{6}
\setcounter{section}{5}
\setcounter{question}{0}

%%%%%%%%%%%%%%%%%%%%%%%%%%%%%%%%%%%%%%%%%%%%%%%%%%%%%%%%%%%%%%%%%%%%%%%%%%%
% Assignment 6.5: Using post-hoc tests to find differences in means
%%%%%%%%%%%%%%%%%%%%%%%%%%%%%%%%%%%%%%%%%%%%%%%%%%%%%%%%%%%%%%%%%%%%%%%%%%%

\rassignment{Using post-hoc tests in R to find differences in means}

The state of Iowa in the USA receives many invoices for services that they buy. In turn, these invoices need to be paid in a timely manner. The questions has been raised whether the state of Iowa pays all invoices equally timely. To investigate this you are requested to perform an audit. In this audit you set out to statistically check if you can find differences -between the various services that are bought- in the time that it takes for Iowa to pay an invoice. \\

For this assignment you will have to download the data file \dataset{iowa.RData} from the online resources\footnote{These data are taken from \url{https://data.iowa.gov/State-Government-Finance/State-of-Iowa-Checkbook/cyqb-8ina}.}. \dataset{.RData} files are compressed \texttt{R} objects, and are useful when dealing with very large data sets such as this one. \\

The \dataset{iowa.RData} file contains payment transactions recorded in the State of Iowa’s central accounting system for the Executive Branch and is real data. \\

\question{
    Load the \dataset{iowa.RData} data file into the environment using the \rcode{load()} function. 
}

\rcodeanswertiny

The data set is now stored in object called \rcode{iowa}. \\

\question{
    Give a short description of the data in \rcode{iowa}.
}

\hint{Search the internet for the source of these data to find out what the columns represent.}

\threelineanswerbox

\question{
    How many rows and columns does the \rcode{iowa} data set have?
}

\emptyanswerbox{
    Rows: \shortanswerline \hspace*{3cm} Columns: \shortanswerline
}

\clearpage % Page break

\question{
    How many unique services are there? Make a \concept{frequency table} of these services.
}

\rcodeanswertiny

\emptyanswerbox{
    Unique services: \shortanswerline 
}

\question{
    Which service has the most rows? How many rows does this service have?
}

\emptyanswerbox{
    Service: \shortanswerline \hspace{3cm} Rows: \shortanswerline
}

\question{
    How many rows show a difference in invoice date and payment date?
}

\rcodeanswertiny

\emptyanswerbox{
    Number of rows that show a difference: \shortanswerline 
}

\question{
    Create a new data set that consists of these differences, and name the new data set \rcode{dataDif}.
}

\rcodeanswermedium

\clearpage % Page break

\question{
    Create an extra column named \rcode{dif.days} in \rcode{dataDif} that contains the number of days between invoice and payment.
}

\hint{Make sure that the column \rcode{dif.days} is numeric.}

\rcodeanswertiny

\question{
    Calculate the \concept{minimum}, \concept{maximum}, \concept{mean}, \concept{quartiles}, and \concept{standard deviation} of the column \rcode{dif.days}.
}

\rcodeanswermedium

\emptyanswerbox{  
    Minimum: \quad \shortanswerline \quad Upper quartile: \quad \quad \shortanswerline
            \answerskip
            Mean: \quad\hspace*{9pt} \shortanswerline \quad \quad Lower quartile: \quad \quad \shortanswerline
            \answerskip
            Maximum: \quad \shortanswerline \quad Standard deviation: \shortanswerline
}

\question{
    Create a histogram of the column \rcode{dif.days}. Describe what you see in the histogram.
}

\rcodeanswertiny

\onelineanswerbox

\clearpage % Page break

\question{
    Again, create a histogram, but now only use the subset of \rcode{dif.days} that is in the 5-95\% quantile range (so you cut off the bottom and top 5\%).
}

\hint{Hint: use the \rcode{quantile()} function.}

\rcodeanswersmall

You don't trust the negative values in \rcode{dif.days} as you cannot interpret them, and therefore you will not include them in your investigation. Moreover, you also don't want to include value in \rcode{dif.days} that are higher than 365 days. \\

\question{
    Create a new data set in which these values are removed and name this data set \rcode{dataDif2}.
}

\rcodeanswertiny

\question{
    Create a scatter plot with \rcode{dif.days} on the \textit{y-axis} and \rcode{Amount} on the \textit{x-axis}.
}

\rcodeanswertiny

\question{
    Compute the \concept{correlation} between the time between invoice and payment, and the amount that is paid.
}

\rcodeanswertiny

\emptyanswerbox{
    Correlation: \shortanswerline 
}

\clearpage % Page break

\question{
    Elaborate on the \concept{correlation coefficient} and it's significance. What does this imply?
}

\threelineanswerbox

\question{
    Compute the \concept{mean} \rcode{dif.days} per expense category.
}

\hint{Use the \rcode{aggregate()} function (for more help on this function see \rcode{?aggregate}).}

\rcodeanswertiny

\twolineanswerbox

\question{
    Use the \rcode{aov()} function to test whether the \concept{means} that you computed in assignment 6.5p are statistically different.
}

\emptyanswerbox{
    p-value: \shortanswerline
    \answerskip
    Conclusion:
    \answerskip
    \rule{\textwidth}{0.4pt}
}

\rcodeanswertiny

\clearpage % Page break

\question{
    Use \concept{Tukey's Honest Significant Differences} to find out which group \concept{means} are truly different.
}

\hint{Use the \rcode{TukeyHSD()} function to find \concept{Tukey's Honest Significant Differences}.} 

\rcodeanswertiny

\twolineanswerbox

In assignment 6.5q you have computed an \concept{ANOVA}, but this is statistically not completely sound. \\

\question{
    Can you formulate why the \concept{ANOVA} in assignment 6.5q was not statistically sound? What would be an appropriate analysis?
}

\twolineanswerbox

\question{
    Why do you think it is a good or bad idea to calculate \concept{p-values} if the number of rows in the data is large?
}

\threelineanswerbox

\clearpage % Page break