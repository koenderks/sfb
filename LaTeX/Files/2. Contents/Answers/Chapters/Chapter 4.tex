\setcounter{section}{4}
\setcounter{subsection}{1}
\setcounter{answer}{0}

\subsection{Chapter 4: Correlation and regression}

\answer{
    Two variables can either be positively related, not related, or negatively related. The most logical relationship is that the distance a customer lives from the store is negatively related to how many times they visit the store.
}

\answer{
    \vspace*{1pt}
    \begin{tabular}{|c|c|r|r|r|r|}
    \hline 
    $i$ & $x_i$ & $y_i$ & $(x_i - \bar{x})$ & $(y_i - \bar{y})$ & $(x_i - \bar{x})(y_i - \bar{y})$ \tstrut\bstrut\\
    \hline
    1 & 4.87 & 2.90 & 1.868 & -1.09 & -2.053 \tstrut\bstrut\\
    \hline
    2 & 3.04 & 4.50 & 0.038 & 0.501 & 0.0194 \tstrut\bstrut\\
    \hline
    3 & 1.65 & 4.94 & -1.351 & 0.941 & -1.272 \tstrut\bstrut\\
    \hline
    4 & 2.88 & 3.28 & -0.121 & -0.718 & 0.087 \tstrut\bstrut\\
    \hline
    5 & 2.31 & 4.73 & 0.691 & 0.731 & -0.505 \tstrut\bstrut\\
    \hline
    6 & 3.96 & 2.64 & 0.958 & -1.358 & -1.303 \tstrut\bstrut\\
    \hline
    7 & 2.70 & 3.70 & -0.301 & -0.298 & 0.089 \tstrut\bstrut\\
    \hline
    8 & 2.60 & 5.30 & -0.401 & 1.301 & -0.522 \tstrut\bstrut\\
    \hline
    \end{tabular}\\
    \\
    \hspace*{.5cm} $\bar{x} = 3.001$ \hspace*{1cm} $\sum (x_i - \bar{x})(y_i - \bar{y}) = -5.458$ \\
    \\
    \hspace*{.5cm} $\bar{y} = 3.998$ \hspace*{3.1cm} $n - 1 = 7$ \\
    \\
    \hspace*{6cm} $s_{xy} = -0.779$ \\
}

\answer{
    The covariance is negative. A negative covariance indicates that that as one variable deviates from the mean, the other variable deviates in the other direction. This means that, when a customer’s distance from the store in kilometers is higher than the mean, their average visits per week will likely be lower than the mean.
}

\answer{
    The disadvantage of using the covariance as a measure for the strength of this relationship is that it depends on the measurement unit (kilometers vs. meters) that the co-worker asks the questions in. If the co-worker would have asked the question in meters the covariance would have increased by a 1000 times, namely -779.84.
}

\answer{
    $s_x = \sqrt{\frac{\sum^n_{i = 1} (x_i - \bar{x})^2}{n - 1}} = \sqrt{\frac{6.977}{7}} = 0.998$ \hspace*{2cm} $s_y = \sqrt{\frac{\sum^n_{i = 1} (y_i - \bar{y})^2}{n - 1}} = \sqrt{\frac{7}{7}} = 1$
}

\answer{
    $r_{xy} = \frac{s_{xy}}{s_x \times s_y} = \frac{-0.779}{0.998 \times 1} = -0.779$
}

\answer{
    The coefficient is -0.779, which represents is a relatively strong negative relationship.
}

\answerbreakline \setcounter{subsection}{2} \setcounter{answer}{0}

\clearpage

\answer{
    $H_0$: $\rho_{xy} \geq 0$ \hspace{4cm} $H_1$: $\rho_{xy} < 0$
}

\answer{
    \begin{minipage}[t]{.5\textwidth}
    $N = $ unknown \\
    $n = 8$ 
    \end{minipage}
    \begin{minipage}[t]{.5\textwidth}
    $r_{xy} = -0.779$ \\
    $\rho_{xy} = $ unknown
    \end{minipage}
}

\answer{
    $z_{0.05} = -1.645$
}

\answer{
    $z_r = \frac{1}{2} \times log_e(\frac{1 + r}{1 - r}) = \frac{1}{2} \times log_e(\frac{1 - 0.779}{1 + 0.779}) = -1.043$ \\
    \\
    $SE_r = \frac{1}{\sqrt{n - 3}} = \frac{1}{\sqrt{8 - 3}} = 0.447$ \\
    \\
    $z_{xy} = \frac{z_r}{SE_r} = \frac{-1.043}{0.447} = -2.33$
}

\answer{
    The observed z-value is more extreme (lower) than the critical z-value. $H_0$ is rejected with 95\% confidence. You can be 95\% confident that $\rho_{xy}$ is negative in the population. There is a 5\% change of a type-I error.
}

\answerbreakline \setcounter{subsection}{3} \setcounter{answer}{0}

\answer{
    \vspace*{-10pt}
    \answercode{{\color{dataset}\# Be sure to set your working directory when providing a relative path} \\
dataset6 <- read.csv(\textquotesingle localSupermarket.csv\textquotesingle)}
}

\answer{
    Covariance: -0.30 \\
    Correlation: -0.30 \\
    \\
     \answercode{cov(dataset6\$Distance, dataset6\$AvgVisits)  {\color{dataset}\# Covariance   -0.3000603} \\
cor(dataset6\$Distance, dataset6\$AvgVisits)  {\color{dataset}\# Correlation: -0.3000382}}   
}

\answer{
    \vspace*{-10pt}
    \answercode{cor.test(dataset6\$Distance, dataset6\$AvgVisits, alternative = \textquotesingle less\textquotesingle) \\
{\color{dataset}\# Correlation: -0.30} \\
{\color{dataset}\# t-value: -9.936} \\
{\color{dataset}\# p-value: < 2.2e-16}}
}

\answer{
    The black line represents the normal distribution. The red line represents the t-distribution. The difference between the two distributions, in terms of their shape, is that the t-distribution has slightly thicker tails. When you increase the degrees of freedom of the t-distribution, it will start to look more like the normal distribution.\\
    \\
    \answercode{curve(dnorm(x, mean = 0, sd = 1), from = -3, to = 3, ylab = \textquotesingle Density\textquotesingle)\\
curve(dt(x, df = 3), from = -3, to = 3, add = TRUE, col = \textquotesingle red\textquotesingle)}
}

\clearpage

\answer{
    $df = 998$ \hspace*{5cm} $t_{xy} = -9.936$ \\
    \\
    \answercode{n <- nrow(dataset6) \\
dft <- n - 2 {\color{dataset}\# 998} \\
 \\
r <- cor(dataset6\$Distance, dataset6\$AvgVisits) \\
tscore <- r * sqrt(n - 2) / sqrt(1 - r\textrm{\textasciicircum}2)    \\
{\color{dataset}\# t-score: -9.936 so you can confirm the value in 4.3c}
}
}

\answer{
    You can find the t-value in the bottom line of the output in the console.
}

\answer{
    The p-value is < 2.2e-16, which is lower than the significance level of 5\%. This means that $H_0$ can be rejected with 95\% confidence.
}

\answerbreakline \setcounter{subsection}{4} \setcounter{answer}{0}

\answer{
    \vspace*{-10pt}
    \answercode{{\color{dataset}\# Be sure to set your working directory when providing a relative path} \\
dataset7 <- read.csv(\textquotesingle nationalSupermarket.csv\textquotesingle)}
}

\answer{
    \vspace*{-10pt}
    \answercode{plot(x = dataset7\$Price, \\
     \hspace*{30pt}y = dataset7\$AvgWasted, \\
     \hspace*{30pt}main = \textquotesingle Scatter plot of Price vs. AvgWasted\textquotesingle, \\
     \hspace*{30pt}ylab = \textquotesingle Average number of cartons wasted\textquotesingle, \\
     \hspace*{30pt}xlab = \textquotesingle Price of a carton of milk\textquotesingle, \\
     \hspace*{30pt}las = 1, \\
     \hspace*{30pt}col = \textquotesingle orange\textquotesingle, \\
     \hspace*{30pt}pch = 19, \\
     \hspace*{30pt}bty = \textquotesingle n\textquotesingle)
}
}

\answer{
    AvgWasted $= \beta_0 + \beta_1 \,\times$ Price
}

\answer{
    \vspace*{-10pt}
    \answercode{lmfit <- lm(formula = AvgWasted {\raise.17ex\hbox{$\scriptstyle\sim$}} Price, data = dataset7)}
}

\answer{
    AvgWasted $= 0.236 + 2.995 \,\times$ Price \\
    \\
    \answercode{summary(lmfit) \\
{\color{dataset}\# b0 = 0.236} \\
{\color{dataset}\# b1 = 2.995} \\
{\color{dataset}\# R-squared: 0.64}
}
}

\clearpage % Page break

\answer{
    \vspace*{-10pt}
    \answercode{abline(lmfit)}
}

\answer{
    $R^2 = 0.64$ \\
    \\
    Interpretation: The multiple $R^2$ is 0.64, meaning that 64\% of the variation in the number of milk cartons that are thrown away each day can be explained by the price of the milk cartons.
}

\answer{
    $H_0$: $\beta_1 \leq 0$ \hspace*{4cm} $H_1$: $\beta_1 > 0$
}

\answer{
    The p-value for the regression coefficient is < 2e-16, which is lower than the significance level of 5\%. $H_0$ can be rejected with 95\% confidence. You can be 95\% sure that $\beta_1$ is positive in the population. The price contributes significantly to the average number of milk cartons thrown away. There is a 5\% risk of a type-I error.
}

\answerbreakline \setcounter{subsection}{5} \setcounter{answer}{0}

\answer{
    \vspace*{-10pt}  
    \answercode{newdata <- data.frame(Price = 0.70)}
}

\answer{
    Predicted value: 2.33 \\
    \\
    \answercode{predict(object = lmfit, \\
        \hspace*{40pt}newdata = newdata) {\color{dataset}\# Prediction: 2.33}}
}

\answer{
    Predicted value: $0.236 + 2.995 \times 0.70 = 2.33$
}

\answer{
    \vspace*{-10pt}
    \answercode{predict(object = lmfit, newdata = newdata, \\
        \hspace*{40pt}interval = \textquotesingle prediction\textquotesingle, level = 0.90) \\
{\color{dataset}\# Lower bound: 0.734} \\
{\color{dataset}\# Upper bound: 3.931}
}
}

\answer{
    The supermarket will throw away fewer cartons of milk. \\
    \\
    \underline{Explanation}: The current number of milk cartons thrown away (4) lies outside the bounds of the 90\% confidence interval for the prediction.
}

\clearpage % Page break