\setcounter{section}{5}
\setcounter{subsection}{3}
\setcounter{question}{0}

%%%%%%%%%%%%%%%%%%%%%%%%%%%%%%%%%%%%%%%%%%%%%%%%%%%%%%%%%%%%%%%%%%%%%%%%%%%
% Assignment 5.3: Independent two-sample t-test by hand and in R
%%%%%%%%%%%%%%%%%%%%%%%%%%%%%%%%%%%%%%%%%%%%%%%%%%%%%%%%%%%%%%%%%%%%%%%%%%%

\handassignment{Independent two-sample t-test by hand and in R}

To test the effect of caffeine on the respiratory exchange ratio (RER), 9 men get caffeine and 9 different men get a placebo, their RER is measured while doing sports\footnote{This example is taken from \url{http://learntech.uwe.ac.uk/da/Default.aspx?pageid=1438}}: \\
\vspace*{0.5cm}
\begin{center}
\begin{tabular}{c|ccc}
RER measurement & $n$ & Mean ($\bar{x}$) & Standard deviation ($s_x$) \tstrut\bstrut\\
\hline
Caffeine & 9 & 94.22 & 6.49 \tstrut\bstrut\\
Placebo & 9 & 101.22 & 8.14 \tstrut\bstrut\\
\end{tabular}
\end{center}
\vspace*{0.5cm}

Researchers want to use this experiment to show with 95\% confidence that caffeine reduces the RER in men. They are going to evaluate the results by comparing the \concept{mean} RER using a \concept{two-sample t-test}. \\

\question{
    Why is this test also called an \concept{independent samples t-test}?
}

\twolineanswerbox

\question{
    Write down the \concept{null hypothesis} $H_0$ and the \concept{alternative hypothesis} $H_1$ for a one-sided test where the researchers want to show that the placebo \concept{mean} RER is higher than the caffeine \concept{mean} RER.
}

\hypothesesbox

In the formula sheet on page~\pageref{formulasheet} you can see that for \concept{independent samples t-tests}:
\vspace*{0.5cm}
\begin{equation*}
    t = \frac{(\bar{x}_1 - \bar{x}_2) - D_0}{\sqrt{s^2_p (\frac{1}{n_1} + \frac{1}{n_2})}}
    \hspace*{3cm}
    s^2_p = \frac{(n_1 - 1) s^2_1 + (n_2 - 1) s^2_2}{n_1 + n_2 - 2}
\end{equation*}
\vspace*{0.5cm}

\question{
    Using the formula provided calculate the \concept{pooled standard deviation} $s_p$.
}

\clearpage % Page break

\emptyanswerbox{
    $s_p$: \shortanswerline
    \answerskip
    Calculation:
    \answerskip
    \rule{\textwidth}{0.4pt}
}

\question{
    Using the formula provided calculate the \concept{t-score}.
}

\emptyanswerbox{
    t-score: \shortanswerline
    \answerskip
    Calculation:
    \answerskip
    \rule{\textwidth}{0.4pt}
}

The \concept{critical t-value} for 16 \concept{degrees of freedom} $(n_1 + n_2 - 2)$ for a one-sided test is 1.746. \\

\question{
    Draw the conclusion for this test. Include the following elements:
        \begin{itemize}
        \item[$\square$] Show how the calculated \concept{t-score} relates to the \concept{critical t-score}.
    \item[$\square$] Discuss whether $H_0$ is rejected or not.
    \item[$\square$] Describe what this tells us about $\mu$ and $\mu_0$.
    \item[$\square$] Describe what type of error is relevant \textit{(type-I or type-II)}.
\end{itemize}
}

\sixlineanswerbox

\clearpage % Page break

\begin{minipage}{0.8\textwidth}
Now you will learn how to do the same independent t-test in \texttt{R}. \\

Run the following code in \texttt{R}: \\
\end{minipage}%
\hfill%
\begin{minipage}{0.1\textwidth}
\includegraphics[width=\linewidth]{Files/Images/displaycode.pdf}
\end{minipage}
\vspace*{.1cm}

\codeblock{{\color{dataset}\# These are the values for the RER test data set}\\
placebo <- c(97, 106, 120, 104, 96, 100, 93, 96, 99)\\
caffeine <- c(97, 92, 95, 100, 95, 88, 85, 106, 89)}

\question{
    Calculate the \concept{mean} and \concept{standard deviation} for the \rcode{placebo} and \rcode{caffeine} variables, confirm that your results match with the previous assignment.
}

\rcodeanswermedium

\emptyanswerbox{
    \rcode{placebo} \hspace*{6cm} \rcode{caffeine}
    \answerskip
    Mean: \hspace{90pt}\tinyanswerline \hspace*{2cm}Mean: \hspace{80pt}\tinyanswerline
    \answerskip
    Standard deviation: \tinyanswerline \hspace*{1.95cm}Standard deviation: \tinyanswerline
}

Run the following code in \texttt{R}: \\

\codeblock{t.test(x = placebo, y = caffeine, alternative = \textquotesingle greater\textquotesingle, mu = 0, \\
       \hspace*{40pt}paired = FALSE, var.equal = TRUE, conf.level = 0.95)}

\question{
    Interpret the \texttt{R} code above and check if the results match your conclusion for assignment 5.3e.
}

\twolineanswerbox

The \concept{t-test} assumes equal \concept{variances}. In this case the \concept{standard deviations} are actually a bit different. \\

\clearpage % Page break

\question{
    Change the \rcode{var.equal} argument in the \texttt{R} code above to do a \concept{Welch t-test} to account for unequal \concept{variances}. Check whether the results differ.
}

\rcodeanswertiny

\twolineanswerbox

\question{
    How can you check if you need to do a \concept{Welch t-test} instead of a normal \concept{t-test}?
}

\twolineanswerbox

\clearpage % Page break