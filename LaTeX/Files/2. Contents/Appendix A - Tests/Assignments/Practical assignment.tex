%%%%%%%%%%%%%%%%%%%%%%%%%%%%%%%%%%%%%%%%%%%%%%%%%%%%%%%%%%%%%%%%%%%%%%%%%%%
% Assessment 1: Practical assignment using R
%%%%%%%%%%%%%%%%%%%%%%%%%%%%%%%%%%%%%%%%%%%%%%%%%%%%%%%%%%%%%%%%%%%%%%%%%%%

\begin{minipage}{0.8\textwidth}
\section{Practical assignment}
\end{minipage}%
\hfill%
\begin{minipage}{0.1\textwidth}
\includegraphics[width=\linewidth]{Files/Images/displaycode.pdf}
\end{minipage}
\vspace*{.1cm}

In this assignment you are asked to choose a data set from the online resources\footnote{The data sets that you can choose from can be found in the folder \textit{Practical assignment}.}, formulate two research questions about the variables in this data set, and explore the data using the knowledge and techniques acquired throughout this workbook. For this, you will need to explore your data set using descriptive statistics and visualization, formulate two hypotheses, perform the two relevant statistical tests and draw conclusions for your research questions. \\

Your first test should contain a comparison of means or proportions performed using the formulas on page~\pageref{formulasheet}, while the second test should contain a linear regression performed using \texttt{R}. \\

\begin{enumerate}
    \item \textbf{(3 points)} Shortly describe the data set you have chosen and formulate two research questions that you will explore. Remember, the first research question should involve a comparison of two means or two proportions, while the second research question should involve a linear relationship between two variables. Write down the business reasoning behind your research questions. 
    \item \textbf{(2 points)} Discuss the variables in your research questions using the appropriate descriptive statistics. Create relevant figures for your research questions and discuss what you can learn from these figures. 
    \item \textbf{(6 points)} Perform a test on your data set using a comparison of means or proportions.
    \begin{enumerate}
        \item[a.] \textbf{(2 points)} Discuss which test you are going to perform and how this relates to the parameters in the first research question. Indicate what parameter(s) you will need to investigate to answer this research question. Next, formulate the appropriate (statistical) null hypothesis $H_0$ and alternative hypothesis $H_1$. Make sure that your hypotheses are formulated correctly.
        \item[b.] \textbf{(2 points)} Calculate the corresponding test statistic (e.g., $t$, $z$) from your data using the formulas on page~\pageref{formulasheet}. Write down the formulas and explain your calculations. 
        \item[c.] \textbf{(2 points)} Define the critical area for your test statistic using Table 2 ($z$-values) on page~\pageref{table2} or Table 3 ($t$-values) on page~\pageref{table3}. Based on the critical value, draw the conclusion about your hypotheses. Explain how you got to this answer.
    \end{enumerate}
    \item \textbf{(6 points)} Perform a test for a linear relationship using linear regression in \texttt{R}.
    \begin{enumerate}
        \item[a.] \textbf{(2 points)} Discuss which test you are going to perform and how this relates to the second research question. Write down the regression equation and indicate what parameter(s) you will need to investigate to answer this research question. Formulate the appropriate (statistical) null and alternative hypothesis. Make sure that your hypotheses are formulated correctly.
        \item[b.] \textbf{(2 points)} Perform the linear regression using \texttt{R}. Provide the relevant \texttt{R} output in your paper and incorporate the relevant statistics in your answer.
        \item[c.] \textbf{(2 points)} Based on your results in \texttt{R}, what do you conclude about your hypotheses? Explain how you got to this answer.
    \end{enumerate}
    \item \textbf{(1 point)} Discuss what your results imply for your population of interest and how they relate to your research questions.
    \item \textbf{(2 points)} Discuss how representative your sample is and which risk applies to your conclusions: \textit{type-I} or \textit{type-II}, and what this implies for your business scenario.
\end{enumerate}

\clearpage % Page break