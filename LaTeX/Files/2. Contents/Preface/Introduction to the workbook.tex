\addtocontents{toc}{\protect\vspace{10pt}}
\section{Introduction to the workbook}

In this introduction some basic principles are defined that are followed throughout the workbook to guide you through discovering statistics using \texttt{R}. Here follows a summation of the different sections that you will encounter while working through the assignments of the various chapters. \\

\bigskip

\textit{Learning objectives} \\

Every chapter has specific learning objectives that give you an initial impression about what you will learn during the assignments of that chapter. These learning objectives also provide a good indication of your ability after finishing the assignments in a chapter, since you can go back and make sure that you master all of them. \\

The learning objectives are shown at the beginning of each chapter in a {\color{learningobjectives} \fontfamily{pcr}\selectfont \textbf{yellow}} box. \\

\emptylearningobjectives \\

\textit{Exercises by hand} \\

\begin{minipage}{0.85\textwidth}
Fancy some mathematical exercise? When you see the icon to the right you know that you will be expected to perform calculations by hand (or by the use of a simple calculator). For example, you may get asked to calculate the mean of a small data set in an assignment.
\end{minipage}%
\hfill%
\begin{minipage}{0.1\textwidth}
\includegraphics[width=\linewidth]{Files/Images/pencilhand.pdf}
\end{minipage} \\
\\
\bigskip

\textit{Exercises in R} \\ 

\begin{minipage}{0.85\textwidth}
The icon to the right is shown when it is time to start programming in \texttt{R}. You may warm up your programming fingers, since you will be asked to write and run your own code in the assignments that follow. 
\end{minipage}%
\hfill%
\begin{minipage}{0.1\textwidth}
\includegraphics[width=\linewidth]{Files/Images/displaycode.pdf}
\end{minipage} \\
\\
\bigskip 

\textit{Hints} \\

\begin{minipage}{0.85\textwidth}
When you see the icon to the right it means that you will be given a hint for that question. Hints can be useful if you do not know how to get started with an assignment, or if you are a little bit lost in the middle of one. Pay close attention to these hints, as they may also guide your attention to important aspects that can be easily overlooked.
\end{minipage}%
\hfill%
\begin{minipage}{0.1\textwidth}
\includegraphics[width=\linewidth]{Files/Images/lightbulb.pdf}
\end{minipage}

\clearpage

\textit{Statistical concepts} \\ 

Various important statistical concepts will present itself during the assignments. These statistical concepts are highlighted in \concept{red} and can be found all over the text of the assignments. When these concepts can be calculated, their formulas can often be found in the formula sheet on page~\pageref{formulasheet}. An example of such an important statistical concept can be the mean. \\

\concept{mean} \\ 

\bigskip

\textit{Data sets} \\

The color \dataset{green} is used to indicate the name of a data set that can be found in the online resources that accompany this workbook. For example, you might be asked to read a data set into \texttt{R} for further use in an assignment. \\

\dataset{example.csv} \\

\bigskip 

\textit{R code} \\ 

\texttt{R} code in the assignments is displayed in \rcode{blue}. These short lines of code are generally used in the assignments to highlight objects or variables that exist in your current \texttt{R} session. It can also highlight certain functions that you can use on an \texttt{R} object. \\

\rcode{View(dataset)} \\ 

\bigskip

\textit{R code blocks} \\ 

\texttt{R} code blocks are used throughout the assignments to indicate \texttt{R} code that should be executed together. Copy this code into your script in \texttt{RStudio} and highlight all of the code to run at it at once. Be sure to do this at all times when you see a code block, as running some lines of code individually may influence your results drastically. \\

\codeblock{x <- c(1, 5, 8, 3) \\ mean(x) }
