%%%%%%%%%%%%%%%%%%%%%%%%%%%%%%%%%%%%%%%%%%%%%%%%%%%%%%%%%%%%%%%%%%%%%%%%%%%
% Grading sheet: Practice exam
%%%%%%%%%%%%%%%%%%%%%%%%%%%%%%%%%%%%%%%%%%%%%%%%%%%%%%%%%%%%%%%%%%%%%%%%%%%

\section{Practice exam}

\fontfamily{pcr}\selectfont 

\underline{\textbf{Question 1 (10 points)}} \\

\begin{itemize}
    \item[\textbf{1a)}] \textbf{(2 points)} \\ 
    \textit{-1 point for every element missing with a minimum of 0 points.}
        \begin{itemize}
        \item[$\blacksquare$] Ordinal        
        \item[$\blacksquare$] Ratio
        \item[$\blacksquare$] Nominal
        \item[$\blacksquare$] Interval 
        \end{itemize}
        
     \item[\textbf{1b)}] \textbf{(3 points)} \\
     \textit{Each correct calculation is worth 1 point.}   \\
     \\
     Mean: $\frac{\sum_{i = 1}^n x_i}{n} = \frac{4 + 2 + 5 + 6 + 2 + 1 + 2 + 5 + 3 + 3 + 7 + 4 + 5 + 1 + 7 + 5 + 3}{17} = \frac{65}{17} = 3.824$\\
     Median: $\frac{n + 1}{2} = 4$th number is 4\\
     Mode: The mode is 5 as it occurs the most (4) times. \\
     \item[\textbf{1c)}] \textbf{(2 points)} \\
     \textit{The answer should contain (-1 point for every element missing with a minimum of 0 points).}
    \begin{itemize}
    \item[$\blacksquare$] The distribution of these values is negatively skewed,    
    \item[$\blacksquare$] since the mean is lower than the median and the median is \\lower than the mode.
    \end{itemize} \\
     \item[\textbf{1d)}] \textbf{(2 points)} \\
     \textit{-1 point for every element missing with a minimum of 0 points.}\\
     \\
     Minimum: 1 \\
     Lower quartile: $\frac{n + 1}{4}$th number is 2\\
     Upper quartile: $\frac{n + 1}{4} \times 3$th number is 5\\
     Maximum: 7 \\
\item[\textbf{1e)}] \textbf{(1 points)} \\
     \textit{1 point is awarded only when all numbers are given correctly.} \\ \\
     a: 1, b: 2, c: 4, d: 5, e: 7 \\
\end{itemize}

\underline{\textbf{Question 2 (10 points)}} \\

\begin{itemize}
\item[\textbf{2a)}] \textbf{(2 points)} \\
     \textit{This can be answered in your own words, but the answer should contain the following elements: (-1 point for every element missing with a \\minimum of 0 points).} \\
    \begin{itemize}
        \item[$\blacksquare$] When a sample is large enough ($n \geq 30$),        
        \item[$\blacksquare$] the sampling distribution,
        \item[$\blacksquare$] is a normal distribution (with a mean equal to the population mean and a standard error equal to / close to $\frac{s}{\sqrt{n}}$).
    \end{itemize}
\item[\textbf{2b)}] \textbf{(2 points)} \\
     \textit{Any answer that shows that the student understands that the (shape / properties of the) normal distribution is used in these tests is correct.} \\
\item[\textbf{2c)}] \textbf{(1 points)} \\ \\
     The Shaprio-Wilk test. \\
\item[\textbf{2d)}] \textbf{(1 points)} \\
     \textit{1 point is awarded only if all three answers are given.} \\
    \begin{itemize}
        \item[$\blacksquare$] Homogeneity of variance (if applicable)
        \item[$\blacksquare$] Data should be measured at least at interval level
        \item[$\blacksquare$] Independence of observations
    \end{itemize}
\item[\textbf{2e)}] \textbf{(2 points)} \\
     \textit{-1 for calculation errors and -1 for wrong $t$-value, no penalty for \\rounding errors.} \\ 
     \\
     For 95\% one-sided confidence and $df = n - 1 = 16 - 1 = 15$ you find $t = 1.753$. The upper bound is: $\bar{x} + t + \frac{s}{\sqrt{n}} = 38 + 1.753 \times \frac{4}{\sqrt{16}} = 39.753$ \\
\item[\textbf{2f)}] \textbf{(2 points)} \\
     \textit{-1 for calculation errors and -1 for wrong $z$-value, no penalty for \\rounding errors.} \\ 
     \\
     For 95\% two-sided confidence interval we need the z-value for a \\left-tailed probability of 97.5\% $\rightarrow$ $z = 1.960$. \\ \\The upper bound is $\bar{x} + z \times \frac{s}{\sqrt{n}} = 250 + 1.960 \times \frac{34}{\sqrt{81}} = 257.41$. \\ \\
     The lower bound is $\bar{x} - z \times \frac{s}{\sqrt{n}} = 250 - 1.960 \times \frac{34}{\sqrt{81}} = 242.59$.
\end{itemize}

\underline{\textbf{Question 3 (15 points)}} \\

\begin{itemize}
\item[\textbf{3a)}] \textbf{(2 points)} \\ \\
$H_0: \mu \geq 10$ \hspace{3cm} $H_1: \mu < 10$ \\
\item[\textbf{3b)}] \textbf{(2 points)} \\
\textit{-1 point if the $z$-value is given as 1.645 instead of -1.645} \\ \\
The critical $z$-value is -1.645. That is because $H_0$ only gets rejected in the case that $\mu < 10$, which should yield a negative $z$-value. \\
\item[\textbf{3b)}] \textbf{(2 points)} \\
\textit{if the $z$-value is calculated as $z = \frac{\bar{x} - \mu}{\sigma} = -1.648$ then -1 (1 point total)} \\ \\
z-value: $z = \frac{\bar{x} - \mu}{\sigma / \sqrt{n}}$ \\ \\
First calculate the population standard deviation: $\sigma = \sqrt{\sigma^2} = 1.517$ \\ \\
z-value: $z = \frac{\bar{x} - \mu}{\sigma / \sqrt{n}} = \frac{7.5 - 10}{1.517 / \sqrt{50}} = -11.65$ \\
\item[\textbf{3d)}] \textbf{(4 points)} \\
\textit{1 point for each included element.}
        \begin{itemize}
        \item[$\blacksquare$] The calculated $z$-value is more extreme (lower) than the critical \\$z$-value.
        \item[$\blacksquare$] This implies that $H_0$ is rejected.
        \item[$\blacksquare$] The mean lead level in the school is lower than the maximum legal amount.
        \item[$\blacksquare$] There is a 5\% probability of a Type-I error.
        \end{itemize} \\
\item[\textbf{3e)}] \textbf{(3 points)} \\
\textit{1 point for each included element. First point is also earned when you have a positive answer for 3b or 3c, but have a good reasoning towards area C.}
        \begin{itemize}
        \item[$\blacksquare$] The p-value is represented by area B,
        \item[$\blacksquare$] since the p-value is the area that is more extreme (lower) than the sample ($z$-)value.
        \item[$\blacksquare$] and represents the probability of observing the sample ($z$-)value or more extreme (lower).
        \end{itemize} \\
\item[\textbf{3f)}] \textbf{(2 points)} \\
\textit{1 point for each included element. As long as this answer is consistent with the conclusion in 3d, full points can be earned.}
        \begin{itemize}
        \item[$\blacksquare$] The p-value is lower than 0.05,
        \item[$\blacksquare$] since the sample $z$-value is more extreme (lower) than the critical $z$-value.
        \end{itemize} \\
\end{itemize}

\underline{\textbf{Question 4 (15 points)}} \\

\begin{itemize}
\item[\textbf{4a)}] \textbf{(11 points)} \\ \\
This is a dependent t-test because the same 10 participants are tested twice. The observations for each test subject are paired / connected / dependent. \textit{(2 points)}\\ \\
$H_0: \mu_0 \leq 0$ \hspace{3cm} $H_1: \mu_0 > 0$ \hspace{3cm} \textit{(2 points)} \\ \\
Where $\mu_0$ is the average of the difference between the endothelial function after eating a high-fat meal and a low-fat meal ($\mu_1 - \mu_2$). \textit{(1 point)} \\ \\ 
\textit{For hypotheses and definition another symbol such as $D_0$ is also correct, using $\mu_1$ and $\mu_2$ is incorrect.} \\ \\
$t = \frac{D}{s_D / \sqrt{n}} = \frac{\bar{x}_1 - \bar{x}_2}{s_D / \sqrt{n}} = \frac{4.9}{6.1 / \sqrt{10}} = 2.540$ \textit{(3 points)}\\ \\
        \begin{itemize}
        \item[$\blacksquare$] The sample $t$-value is 2.540 which is higher than the critical $t$-value of 2.262.
        \item[$\blacksquare$] $H_0$ is rejected.
        \item[$\blacksquare$] The mean endothelial function is significantly lower after eating a high-fat meal than after eating a low-fat meal.
\item[$\blacksquare$] There is a risk of 5\% of a Type I error.
        \end{itemize}
        \textit{Correct evaluation of t-values to reject $H_0$ (2 points) \\ Conclusion parts 3 and 4 (1 point)} \\
\item[\textbf{4b)}] \textbf{(1 point)} \\ \\
You can use Welch’s t-test. \\
\item[\textbf{4c)}] \textbf{(3 points)} \\
        \begin{itemize}
        \item[$\blacksquare$] Homogeneity of variance is only relevant when the variance can \\differ different categories, groups or variables. \textit{(1 point)}
        \item[$\blacksquare$] In a dependent two-sample t-test the variable is the difference \\between the two paired observations, there is therefore only one \\variable with one variance and homogeneity of variance is \\therefore not relevant. \textit{(2 points)}
        \end{itemize} \\
\end{itemize}

\underline{\textbf{Question 5 (15 points)}} \\

\begin{itemize}
\item[\textbf{5a)}] \textbf{(3 points)} \\
\textit{-1 point for every regression term ($\beta_0,\, \beta_1 \times \text{sugar},\, \beta_2 \times \text{vitamins},\, \epsilon$ or error) that is incorrect. \\-1 point if the error term is forgotten.} \\ \\ 
$\text{price}_i = \beta_0 + \beta_1 \times \text{sugar} + \beta_2 \times \text{vitamins} + \epsilon\, (error)$ \\
\item[\textbf{5b)}] \textbf{(1 point)} \\
\textit{1 point also awarded if answer is already given in 5a.} \\ \\ 
$\text{price}_i = 0.60 + 0.15 \times \text{sugar} + 0.10 \times \text{vitamins}$ \\
\item[\textbf{5c)}] \textbf{(1 point)} \\ \\ 
$R^2 = \frac{SS_M}{SS_T} = \frac{200}{250} = 0.8$ \\
\item[\textbf{5d)}] \textbf{(1 point)} \\ \\ 
The amount of sugar and vitamins explain 80\% of the variation in the price of the baby-food. \\ \\ 
or \\ \\ 
The predictor variables in the linear model explain 80\% of the variation in the outcome variable. \\
\item[\textbf{5e)}] \textbf{(1 point)} \\ \\
The problem with the $R^2$ to compare several models is that $R^2$ it always \\increases when you add more predictor variables to the model. \textit{(2 points)} \\
We can use the AIC (or adjusted $R^2$) statistic to reliably compare two linear models. \textit{(1 point)} \\
\item[\textbf{5f)}] \textbf{(1 point)} \\ \\
These variables are called dummy variables. \\
\item[\textbf{5g)}] \textbf{(1 point)} \\ \\
This linear model is equivalent to an ANCOVA because there are two dummy predictors and two continuous predictors. \\
\item[\textbf{5h)}] \textbf{(4 points)} \\ \\
$F = \frac{MS_M}{MS_R}$ \\ \\
$df_M = k-1=3-1=2$ \\
$Ms_M = \frac{SS_M}{df_M} = \frac{200}{2} = 100$ \\ \\
$df_R = n - k = 50 - 3 = 47$ \\
$MS_R = \frac{SS_R}{df_R} = \frac{50}{47} = 1.064$ \\ \\ 
$F = \frac{MS_M}{MS_R} = \frac{100}{1.064} = 93.985$ \\ \\
\textit{Last step: 1 point for correct $F$ formula use and 1 point for correct $F$ input and outcome; \\i.e. max 1 point for $F$ in case of calculation errors.} \\
\end{itemize}

\underline{\textbf{Question 6 (10 points)}} \\

\begin{itemize}
\item[\textbf{6a)}] \textbf{(8 points)} \\ \\
$H_0:$ The 2019 distribution is equal to the historical distribution. \\
$H_1$: The 2019 distribution is not equal to the historical distribution. \\ \textit{(2 points)} \\ \\
Calculation: \\ \\
Expected frequency $E = \text{historical \%} \times n$ \\
    \begin{center}
    \begin{tabular}{|l|c|c|c|c|c|}
    \cline{2-6}
     \multicolumn{1}{c|}{} & \textbf{September} & \textbf{October} & \textbf{November} & \textbf{December} & \textbf{Total} \tstrut\bstrut\\
    \hline
    Observed ($O$) & 102 & 126 & 113 & 88 & 429 \tstrut\bstrut\\
    \hline
    Expected ($E$) & 101.67 & 112.83 & 108.97 & 105.53 & 429 \tstrut\bstrut\\
    \hline
    Deviation ($O - E$) & 0.327 & 13.173 & 4.034 & -17.534 & \multicolumn{1}{|c}{} \tstrut\bstrut\\
    \noalign{\hrule height 2pt}
    $X^2 = \sum \frac{(O - E)^2}{E}$ & 0.0011 & 1.538 & 0.1493 & 2.913 & 4.602 \tstrut\bstrut\\
    \hline
    \end{tabular}
\end{center}
        \begin{itemize}
        \item[$\blacksquare$] Correct calculation of the expected frequency ($E$). \textit{(1 point)}
        \item[$\blacksquare$] Correct calculation of the deviations ($O - E$). \textit{(1 point)}
        \item[$\blacksquare$] Correct calculation of Chi-squared ($X^2$) statistic. \textit{(1 point)}
    \end{itemize} \\
    Conclusion: 
    \begin{itemize}
        \item[$\blacksquare$] The calculated Chi-squared value of 4.602 is lower than the \\critical Chi-squared value of 6.251.
        \item[$\blacksquare$] $H_0$ is not rejected.
        \item[$\blacksquare$] The 2019 rainfall distribution is not significantly different from the historical distribution 
        \item[$\blacksquare$] There is a risk for a Type-II error.
    \end{itemize} \\
    \textit{Correct evaluation of Chi-squared to not reject $H_0$ (2 points) \\ Conclusion parts 3 and 4 (1 point.)} \\
\item[\textbf{6b)}] \textbf{(2 points)} \\
    \begin{itemize}
        \item[$\blacksquare$] There is a minimum of 5 for the expected frequency in all categories.\\ \textit{(1 point)}
        \item[$\blacksquare$] You can use Fisher’s exact test if this requirement is not met.\\ \textit{(1 point)}
        \end{itemize}
\end{itemize}

\underline{\textbf{Question 7 (15 points)}} \\

\begin{itemize}
\item[\textbf{7a)}] \textbf{(6 points)} \\ \\ 
Covariance: $s_{xy} = \frac{\sum_{i=1}^n (x_i - \bar{x}) (y_i - \bar{y})}{n - 1}$
 \begin{center}
    \begin{tabular}{|c|c|c|c|c|c|}
    \hline
    i & $x_i$ & $y_i$ & $(x_i - \bar{x})$ & $(y_i - \bar{y})$ & $(x_i - \bar{x}) (y_i - \bar{y})$ \tstrut\bstrut\\
    \hline
    1 & 100 & 40 & -29.167 & -10.833 & 315.972 \tstrut\bstrut\\
    \hline
    2 & 120 & 60 & -9.167 & 9.1667 & -84.028 \tstrut\bstrut\\
    \hline
    3 & 125 & 45 & -4.167 & -5.833 & 24.306 \tstrut\bstrut\\
    \hline
    4 & 110 & 30 & -19.167 & -20.833 & 399.306 \tstrut\bstrut\\
    \hline
    5 & 170 & 70 & 40.833 & 19.167 & 782.639 \tstrut\bstrut\\
    \hline
    6 & 150 & 60 & 20.833 & 9.167 & 190.972 \tstrut\bstrut\\
    \noalign{\hrule height 2pt}
    \multicolumn{1}{c|}{} & $\bar{x} = 129.167$ & $\bar{y} = 50.833$ & \multicolumn{2}{c|}{} & $\frac{\sum_{i=1}^n (x_i - \bar{x}) (y_i - \bar{y})}{n - 1} = 1629.167$ \tstrut\bstrut\\
    \cline{2-3} \cline{6-6}
    \end{tabular}
\end{center}
Covariance: $s_{xy} = \frac{\sum_{i=1}^n (x_i - \bar{x}) (y_i - \bar{y})}{n - 1} = \frac{1629.167}{5} = 325.833$ \\
        \begin{itemize}
        \item[$\blacksquare$] Correct calculation of $\bar{x}$ \hspace{2.2cm} \textit{(1 point)}
        \item[$\blacksquare$] Correct calculation of $\bar{y}$ \hspace{2.2cm} \textit{(1 point)}
        \item[$\blacksquare$] Correct calculation of $(x_i - \bar{x})$ \hspace{1.1cm} \textit{(1 point)}
        \item[$\blacksquare$] Correct calculation of $(y_i - \bar{y})$ \hspace{1.1cm} \textit{(1 point)}
        \item[$\blacksquare$] Correct calculation of $(x_i - \bar{x}) (y_i - \bar{y})$ \textit{(1 point)}
        \item[$\blacksquare$] Correct calculation of $s_{xy}$ \hspace{1.9cm} \textit{(1 point)}
    \end{itemize} \\
\item[\textbf{7b)}] \textbf{(5 points)} \\ \\ 
Correlation: $r_{xy} = \frac{s_{xy}}{s_x \times s_y}$ \\ \\
First, calculate the standard deviations $s_x$ and $s_y$.\\ \\
Standard deviation: $s = \sqrt{\frac{\sum_{i=1}^n (x_i - \bar{x})^2}{n - 1}}$ \\ \\
Calculation:
 \begin{center}
    \begin{tabular}{|c|c|c|c|c|}
    \hline
    $i$ & $(x_i - \bar{x})$ & $(x_i - \bar{x})^2$ & $(y_i - \bar{y})$ & $(y_i - \bar{y})^2$ \tstrut\bstrut\\
    \hline
    1 & -29.167 & 850.695 & -10.833 & 117.361 \tstrut\bstrut\\
    \hline
    2 & -9.167 & 84.028 & 9.1667 & 84.028 \tstrut\bstrut\\
    \hline
    3 & -4.167 & 17.361 & -5.833 & 34.028 \tstrut\bstrut\\
    \hline
    4 & -19.167 & 367.361 & -20.833 & 434.028 \tstrut\bstrut\\
    \hline
    5 & 40.833 & 1667.361 & 19.167 & 367.361 \tstrut\bstrut\\
    \hline
    6 & 20.833 & 434.028 & 9.167 & 84.028 \tstrut\bstrut\\
    \noalign{\hrule height 2pt}
    \multicolumn{2}{c|}{} & $\sum_{i=1}^n (x_i - \bar{x})^2 = 3420.833$ & & $\sum_{i=1}^n (y_i - \bar{y})^2 = 1120.833$ \tstrut\bstrut\\
    \cline{3-3} \cline{5-5}
    \end{tabular}
\end{center} \\
$s_x = \sqrt{\frac{\sum_{i=1}^n (x_i - \bar{x})^2}{n - 1}} = \frac{3420.833}{5} = 26.157$ \hspace{1cm} 
$s_y = \sqrt{\frac{\sum_{i=1}^n (y_i - \bar{y})^2}{n - 1}} = \frac{1120.833}{5} = 14.972$ \\ \\ 
$r_{xy} = \frac{s_{xy}}{s_x \times s_y} = \frac{325.833}{26.157 \times 14.972} = 0.832$ \\ 
        \begin{itemize}
        \item[$\blacksquare$] Correct calculation of $(x_i - \bar{x})^2$ \hspace{1cm} \textit{(1 point)}
        \item[$\blacksquare$] Correct calculation of $(y_i - \bar{y})^2$ \hspace{1cm} \textit{(1 point)}
        \item[$\blacksquare$] Correct calculation of $s_x$ \hspace{2.1cm} \textit{(1 point)}
        \item[$\blacksquare$] Correct calculation of $s_y$ \hspace{2.1cm} \textit{(1 point)}
        \item[$\blacksquare$] Correct calculation of $r_{xy}$ \hspace{2cm} \textit{(1 point)}
    \end{itemize} \\
\item[\textbf{7c)}] \textbf{(1 point)} \\
\textit{The use of 'r' or the word 'correlation' is incorrect.} \\ \\ 
$H_0: \rho_{xy} \leq 0$ \hspace{3cm} $H_1: \rho_{xy} > 0$ \\
\item[\textbf{7d)}] \textbf{(1 point)} \\
\textit{1 point also awarded for correct formula use with an incorrect \\correlation from question 7b.} \\ \\ 
$t = \frac{r \sqrt{n - 2}}{\sqrt{1 - r^2}} = \frac{0.832 \sqrt{6 - 2}}{\sqrt{1 - 0.832^2}} = 2.999$ \\
\item[\textbf{7e)}] \textbf{(2 points)} \\
        \begin{itemize}
        \item[$\blacksquare$] The calculated t-value is more extreme (higher) than the critical $t$-value. \textit{(1 point)}
        \item[$\blacksquare$] This implies that $H_0$ is rejected. \textit{(1 point)}
        \end{itemize}
\end{itemize}

\normalfont

\clearpage % Page break