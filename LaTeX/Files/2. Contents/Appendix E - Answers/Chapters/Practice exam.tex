%%%%%%%%%%%%%%%%%%%%%%%%%%%%%%%%%%%%%%%%%%%%%%%%%%%%%%%%%%%%%%%%%%%%%%%%%%%
% Grading sheet: Practice exam
%%%%%%%%%%%%%%%%%%%%%%%%%%%%%%%%%%%%%%%%%%%%%%%%%%%%%%%%%%%%%%%%%%%%%%%%%%%

\section{Practice exam}

\fontfamily{pcr}\selectfont 

\underline{\textbf{Question 1 (10 points)}} \\

\begin{itemize}
    \item[\textbf{1a)}] \textbf{(2 points)} \\ 
    \textit{-1 point for every element missing with a minimum of 0 points.}
        \begin{itemize}
        \item[$\blacksquare$] Ordinal        
        \item[$\blacksquare$] Ratio
        \item[$\blacksquare$] Nominal
        \item[$\blacksquare$] Interval 
        \end{itemize}
        
     \item[\textbf{1b)}] \textbf{(3 points)} \\
     \textit{Each correct calculation is worth 1 point.}   \\
     \\
     Mean: $\frac{\sum_{i = 1}^n x_i}{n} = \frac{4 + 2 + 5 + 6 + 2 + 1 + 2 + 5 + 3 + 3 + 7 + 4 + 5 + 1 + 7 + 5 + 3}{17} = \frac{65}{17} = 3.824$\\
     Median: $\frac{n + 1}{2} = 4$th number is 4\\
     Mode: The mode is 5 as it occurs the most (4) times.
     
     \item[\textbf{1c)}] \textbf{(2 points)} \\
     \textit{The answer should contain (-1 point for every element missing with a minimum of 0 points):}
    \begin{itemize}
    \item[$\blacksquare$] The distribution of these values is negatively skewed.        
    \item[$\blacksquare$] Since the mean is lower than the median, and the median is 
lower than the mode.
    \end{itemize}
     
     \item[\textbf{1d)}] \textbf{(2 points)} \\
     \textit{-1 point for every element missing with a minimum of 0 points.}\\
     \\
     Minimum: 1 \\
     Lower quartile: $\frac{n + 1}{4}$th number is 2\\
     Upper quartile: $\frac{n + 1}{4} \times 3$th number is 5\\
     Maximum: 7
    
\end{itemize}

\normalfont

\clearpage % Page break