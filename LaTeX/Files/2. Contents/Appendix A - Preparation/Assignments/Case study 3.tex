%%%%%%%%%%%%%%%%%%%%%%%%%%%%%%%%%%%%%%%%%%%%%%%%%%%%%%%%%%%%%%%%%%%%%%%%%%%
% Case study 3: ...
%%%%%%%%%%%%%%%%%%%%%%%%%%%%%%%%%%%%%%%%%%%%%%%%%%%%%%%%%%%%%%%%%%%%%%%%%%%

\begin{minipage}{0.8\textwidth}
\section{Case study 3: Estimating the total misstatement in an audit}
\end{minipage}%
\hfill%
\begin{minipage}{0.1\textwidth}
\includegraphics[width=\linewidth]{Files/Images/lettericon.pdf}
\end{minipage}
\vspace*{.1cm}

In this case study you may work in groups (max. 4 students) to perform a statistical analysis in \texttt{R}, and write a small report about the results in the form of a management note. For this case study you will need the data set \dataset{BuildIt.csv} from the online resources. The case you will be working on this week is the following: \\

\textit{You and your team work at a Big 4 accountancy firm during an audit of a construction company: BuildIt. The objective of the audit is to determine the extent to which BuildIt's financial statements are presented in a true and fair way. To make a statement about this for the inventory account, your team must take a sample of all stored goods that are recorded in the financial statements, count the value of these goods, and write down their true value (audit value). Next, you investigate the difference between the recorded value of the sample in the financial statements and their audit values. This allows you to estimate the total misstatement (error) in the population. According to industry-specific demands, the maximum allowed misstatement is five percent of the total value of the population. The client is specifically interested in whether they comply with these industry demands.} \\

Write a small (less than two page) report to the client's management where you provide your insights and recommendations. The report should describe your analysis and how you arrived at your conclusions. It should at least contain the following points: \\

\begin{itemize}
    \item[$\blacksquare$] A short introduction about the topic of the report and why it is relevant for the client.
    \item[$\blacksquare$] A description of each variable in the data set, including what it represents and what measurement level it is.
    \item[$\blacksquare$] An explanation of how many samples you intend to select, and how you selected this sample.
    \item[$\blacksquare$] A calculation where you estimate the mean difference between the book values and the audit values using a confidence interval with 95\% certainty.
    \item[$\blacksquare$] A clear explanation of the interpretation of this confidence interval.
    \item[$\blacksquare$] Your opinion about the size of the chosen sample and its effect on the precision of the confidence interval. You can take more samples if you deem it necessary.
    \item[$\blacksquare$] On the basis of the results, your answer to the question(s) posed by the client and an explanation of how you arrived at this answer.
\end{itemize}

\clearpage % Page break