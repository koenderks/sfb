%%%%%%%%%%%%%%%%%%%%%%%%%%%%%%%%%%%%%%%%%%%%%%%%%%%%%%%%%%%%%%%%%%%%%%%%%%%
% Case study 3: ...
%%%%%%%%%%%%%%%%%%%%%%%%%%%%%%%%%%%%%%%%%%%%%%%%%%%%%%%%%%%%%%%%%%%%%%%%%%%

\begin{minipage}{0.8\textwidth}
\section{Case study 3: Estimating the total misstatement in an audit}
\end{minipage}%
\hfill%
\begin{minipage}{0.1\textwidth}
\includegraphics[width=\linewidth]{Files/Images/lettericon.pdf}
\end{minipage}
\vspace*{.1cm}

In this case study you can work in groups (max. 4 students) to perform a statistical analysis in \texttt{R}, write a small report, and present the results to your client. For this case study you will need the data set \dataset{BuildIt.csv} from the online resources. The case you will be working on this week is the following: \\

Your team works in an audit of a construction company BuildIt and has to determine to which extent the financial statements of the company are presented in a true and fair way. To make a statement about the total misstatement in the financial statements, your teams must take a sample recorded transactions in the financial statements, audit these transactions, and write down their true value (audit value). Next, you investigate the difference between the book value and the audit value in the sample to estimate the total error in the population. According to industry-specific demands, the maximum allowed misstatement is five percent of the total value of the population. The client is interested in whether they comply with these industry demands. \\

Write a small report to the client's management where you give your insights and recommendations. The report should describe your analysis and how you arrived at your conclusions. It should at least contain the following points: \\

\begin{itemize}
    \item[$\blacksquare$] A short introduction about the reason and topic of the report.
    \item[$\blacksquare$] What the variables in the data set represent on a practical level and what their measurement level is. 
    \item[$\blacksquare$] An explanation of the chosen sample size and how you selected the sample.
    \item[$\blacksquare$] A calculation where you estimate the mean difference between the book values and the audit values using a confidence interval.
    \item[$\blacksquare$] A clear explanation of the interpretation of this confidence interval.
    \item[$\blacksquare$] Your opinion about the size of the chosen sample and its effect on the precision of the confidence interval. 
    \item[$\blacksquare$] On the basis of these results, your answer to the question(s) posed by the client's management.
\end{itemize}

\clearpage % Page break