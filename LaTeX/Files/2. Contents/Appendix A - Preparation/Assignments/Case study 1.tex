%%%%%%%%%%%%%%%%%%%%%%%%%%%%%%%%%%%%%%%%%%%%%%%%%%%%%%%%%%%%%%%%%%%%%%%%%%%
% Case study 1: Insight into consumer populations
%%%%%%%%%%%%%%%%%%%%%%%%%%%%%%%%%%%%%%%%%%%%%%%%%%%%%%%%%%%%%%%%%%%%%%%%%%%

\begin{minipage}{0.8\textwidth}
\section{Case study 1: Providing insight into customer populations}
\end{minipage}%
\hfill%
\begin{minipage}{0.1\textwidth}
\includegraphics[width=\linewidth]{Files/Images/lettericon.pdf}
\end{minipage}
\vspace*{.1cm}

In this case study you may work in groups (max. 4 students) to perform a statistical analysis in \texttt{R}, and write a small report about the results in the form of a management note. For this case study you will need the data set \dataset{shaving.csv} from the online resources. The case you will be working on this week is the following: \\

\textit{You and your team are working in the advisory department of a Big 4 Accounting firm, whom is hired by a client to provide insight about their customer population. The client in question is a company that produces shaving products, and they have already asked a random sample of 300 customers to provide a rating of their flagship product. Other than giving a rating of the product, the respondents were also asked about some demographics, such as their highest education and in what sector they are currently employed. The client specifically wants to know type of people rate their product the highest, so that they can target those people for advertising.} \\

Write a small (less than two page) report to the client's management where you provide your insights and recommendations. The report should describe your analysis and how you arrived at your conclusions. It should at least contain the following points: \\

\begin{itemize}
    \item[$\blacksquare$] A short introduction about the topic of the report and why it is relevant for the client.
    \item[$\blacksquare$] A description of each variable in the data set, including what it represents and what measurement level it is. 
    \item[$\blacksquare$] A table containing the descriptive statistics for the rating in the sample of customers, split by the respondents' gender. Report, only where meaningful, the frequencies or means for each variable. Discuss what you can learn from this table. 
    \item[$\blacksquare$] For the highest rating gender, a table containing the descriptive statistics for the rating in the sample of customers, split by education level. Discuss what you can learn from this table. 
    \item[$\blacksquare$] For the highest rating gender and education combination, a table containing the descriptive statistics for the rating in the sample of customers, split by job sector. Discuss what you can learn from this table. 
    \item[$\blacksquare$] For the highest rating gender, education, and job combination, report the mean, median, and mode of the rating. Using these measures, describe the distribution of the rating in this particular category of customers. Discuss what observations you can make from this distribution with respect to skewness. 
    \item[$\blacksquare$] An argumentation about whether you think this sample is reliable enough (or not) to make a good decision about all the customers of the client.
    \item[$\blacksquare$] On the basis of the results, your answer to the question(s) posed by the client and an explanation of how you arrived at this answer.
\end{itemize}

\clearpage % Page break