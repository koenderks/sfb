%%%%%%%%%%%%%%%%%%%%%%%%%%%%%%%%%%%%%%%%%%%%%%%%%%%%%%%%%%%%%%%%%%%%%%%%%%%
% Assessment 1: Practical assignment using R
%%%%%%%%%%%%%%%%%%%%%%%%%%%%%%%%%%%%%%%%%%%%%%%%%%%%%%%%%%%%%%%%%%%%%%%%%%%

\begin{minipage}{0.8\textwidth}
\section{Practical assignment}
\end{minipage}%
\hfill%
\begin{minipage}{0.1\textwidth}
\includegraphics[width=\linewidth]{Files/Images/lettericon.pdf}
\end{minipage}
\vspace*{.1cm}

{\footnotesize

In this individual assignment you are asked to investigate a data set, formulate one or more research questions, and explore these questions using the knowledge and techniques you acquired in this course. More concretely, you will need to explore these data using descriptive statistics and visualization, investigate two research questions, perform the appropriate statistical tests and draw sound conclusions. The first test should contain a comparison of means or proportions performed using the formulas on page~\pageref{formulasheet}, the second test should contain a linear regression performed using \texttt{R}. There are 5 questions for a total of 20 points. Your assignment should be in essay form and should describe your analysis. All included tables should be formatted in \textit{APA} style. \\

The scenario is as follows: you are a data analyst at a small supermarket chain that has recently been publicly criticised for its high work pressure. Management has ordered an internal investigation and has asked 300 out of the 500 employees at the firm to fill in a questionnaire about their experiences with high work pressure. You are given the task to find out what factors underlie this high level of stress amongst employees. \\

The data set that you will investigate is unique, and can be found in Canvas as your student number (e.g., \dataset{1020183.csv}). \\

\begin{enumerate}
    \item \textbf{(3 points)} Discuss to what extent there is a difference or a relation between the variables \textit{Gender}, \textit{Age}, and \textit{Stress} in your data set using the appropriate descriptive statistics. 
    \begin{itemize}
        \item[a.] \textbf{(1 point)} Create a table that provides the sample size, the mean, the variance, and standard deviation for the \textit{Age} and \textit{Stress} level of the group of \textit{Males} and the group of \textit{Females} in the data set. Include this table in your report and discuss what you can learn from it. 
        \item[b.] \textbf{(1 point)} Create two visualizations that provide information about the spread within these groups with respect to these two variables. Include these visualizations in your report and discuss what you can learn from it.  
        \item[c.] \textbf{(1 point)} Discuss the representativeness of this sample and the validity of your conclusions when making decisions about whether the high stress levels are caused by an \textit{Age} or \textit{Gender} differences on the basis of this information.
    \end{itemize}
    \item \textbf{(3 points)} Formulate two research question(s) that you will explore using the two required statistical tests to find out what causes the high stress levels among employees. Use new variables in each research question.
    \begin{itemize}
        \item[a.] \textbf{(1 point)} Create a plot that has all the variables plotted against each other to discover possible relations in the data set, and discuss what you can learn from it.
        \item[b.] \textbf{(1 point)} Write down the reasoning behind your research question(s). This reasoning should be based on theoretical and visual (from the figure) information.
        \item[c.] \textbf{(1 point)} Indicate what variable(s) and statistical parameter(s) you will investigate in your research questions.
    \end{itemize}
    \item \textbf{(6 points)} Test your first research question using a test of means or a test of proportions.
    \begin{itemize}
        \item[a.] \textbf{(2 points)} Discuss the test you are going to perform and how this helps you determine factors that may contribute to the high stress levels. Formulate the appropriate (statistical) null and alternative hypothesis. Make sure that your hypotheses are formulated in terms of the statistical parameters you formulated earlier. 
        \item[b.] \textbf{(2 points)} Calculate the corresponding test statistic (e.g., $t$, $z$, $F$, $X^2$) from your data using the formulas learned in this course. Include the full calculation (including filled in formulas) in your report. You may use \texttt{R} to perform the calculations.
        \item[c.] \textbf{(2 points)} Define the critical area for your test statistic using the formulas learned in this course. Based on your previous calculations, draw a sound four-part conclusion. Argue how you got to this answer.
    \end{itemize}
    \item \textbf{(6 points)} Test your second research question using linear regression.
    \begin{itemize}
        \item[a.] \textbf{(2 points)} Discuss the test you are going to perform and how this helps you determine factors that may contribute to the high stress levels. Formulate the appropriate (statistical) null and alternative hypothesis. Make sure that your hypotheses are formulated in terms of the statistical parameters you formulated earlier.
        \item[b.] \textbf{(2 points)} Perform the corresponding linear regression using \texttt{R}. Provide the relevant output in your paper and discuss what you can learn from it. 
        \item[c.] \textbf{(2 points)} Based on the results of your analysis in \texttt{R}, formulate a sound four-part conclusion about your hypotheses? Argue how you got to this answer and discuss the relevant statistics that informed your decision. 
    \end{itemize}
    \item \textbf{(2 points)} Discuss what risks are associated with making decisions on the basis of your performed tests, and what this means for your firm’s management business scenario.
\end{enumerate}

}
\clearpage % Page break