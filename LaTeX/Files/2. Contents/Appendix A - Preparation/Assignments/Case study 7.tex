%%%%%%%%%%%%%%%%%%%%%%%%%%%%%%%%%%%%%%%%%%%%%%%%%%%%%%%%%%%%%%%%%%%%%%%%%%%
% Case study 7: Assessing financial fraud using Benford's law
%%%%%%%%%%%%%%%%%%%%%%%%%%%%%%%%%%%%%%%%%%%%%%%%%%%%%%%%%%%%%%%%%%%%%%%%%%%

\begin{minipage}{0.8\textwidth}
\section{Case study 7: Assessing potential data tampering using Benford's law}
\end{minipage}%
\hfill%
\begin{minipage}{0.1\textwidth}
\includegraphics[width=\linewidth]{Files/Images/lettericon.pdf}
\end{minipage}
\vspace*{.1cm}

In this case study you may work in groups (max. 4 students) to perform a statistical analysis in \texttt{R}, and write a small report about the results in the form of a management note. For this case study you will need the data set \dataset{sinoForest.csv}\footnote{This data set is featured in the \texttt{R} package \texttt{benford.analysis}.} from the online resources. The case you will be working on this week is the following: \\

\textit{You and your team are part of the audit division at an international accounting firm. Your division is called upon to perform an audit of the Sino-Forest corporation, (which was) the leading commercial forest plantation operator in China\footnote{The Sino-Forest corporation was accused of fraud in 2011, and on July 13, 2017, it was decided that four individuals at the corporation had indeed committed fraud.}. Your client is specifically interested in knowing whether the financial numbers are likely to be tampered with. To determine whether this tampering has likely occurred, your team decides to apply the principles of \href{https://en.wikipedia.org/wiki/Benford\%27s_law}{Benford's law} to the (first numbers in the) data set.} \\

Write a small (less than two page) report to the client's management where you provide your insights and recommendations. The report should describe your analysis and how you arrived at your conclusions. It should at least contain the following points: \\

\begin{itemize}
    \item[$\blacksquare$] A short introduction about the topic of the report and why it is relevant for the client.
    \item[$\blacksquare$] A short explanation of what Benford's law represents and how it can be tested. 
    \item[$\blacksquare$] A description of each variable in the data set, including what it represents and what measurement level it is. 
    \item[$\blacksquare$] A table containing the frequency of the first digits in the data set, alongside a table containing the expected frequency of the first digits under Benford's law.
    \item[$\blacksquare$] A statistical formulation of the hypotheses that you intent to investigate, and what parameters are involved.
    \item[$\blacksquare$] A test of the frequencies of the first digits in the data against their expected frequencies under Benford's law.
    \item[$\blacksquare$] On the basis of the results, your answer to the question(s) posed by the client and an explanation of how you arrived at this answer.
\end{itemize}

\clearpage % Page break