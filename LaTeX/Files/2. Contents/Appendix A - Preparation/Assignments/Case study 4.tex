%%%%%%%%%%%%%%%%%%%%%%%%%%%%%%%%%%%%%%%%%%%%%%%%%%%%%%%%%%%%%%%%%%%%%%%%%%%
% Case study 4: ...
%%%%%%%%%%%%%%%%%%%%%%%%%%%%%%%%%%%%%%%%%%%%%%%%%%%%%%%%%%%%%%%%%%%%%%%%%%%

\begin{minipage}{0.8\textwidth}
\section{Case study 4: Determining predictors of credit rating}
\end{minipage}%
\hfill%
\begin{minipage}{0.1\textwidth}
\includegraphics[width=\linewidth]{Files/Images/lettericon.pdf}
\end{minipage}
\vspace*{.1cm}

In this case study you can work in groups (max. 4 students) to perform a statistical analysis in \texttt{R}, write a small report, and present the results to your client. For this case study you will need the data set \dataset{creditRating.csv} from the online resources. The case you will be working on this week is the following: \\

\textit{Your team works for a large bank and is in charge of designing an algorithm that automatically assigns people a credit rating, indicating whether they are likely to default on a loan they received from the bank. To predict this credit rating, management has asked of your team to find out what factors should be considered in the prediction.} \\

Write a small report to the client's management where you give your insights and recommendations. The report should describe your analysis and how you arrived at your conclusions. It should at least contain the following points: \\

\begin{itemize}
    \item[$\blacksquare$] A short introduction about the reason and topic of the report.
    \item[$\blacksquare$] What the variables in the data set represent on a practical level and what their measurement level is. 
    \item[$\blacksquare$] A visualization of the linear relationships in the data set, providing an indication of what to look for. 
    \item[$\blacksquare$] A linear regression predicting the credit rating of a person using all other sensible variables in the data set.
    \item[$\blacksquare$] A check of the assumptions of your linear regression analysis and a statement of whether these are violated.
    \item[$\blacksquare$] On the basis of your results, the answer to the question(s) posed by the client's management and an explanation of how you arrived at this answer.
\end{itemize}

\clearpage % Page break