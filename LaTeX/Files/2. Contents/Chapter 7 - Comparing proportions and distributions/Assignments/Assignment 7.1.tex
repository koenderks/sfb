\setcounter{section}{7}
\setcounter{subsection}{1}
\setcounter{question}{0}
\setcounter{hint}{0} % Only for first assignment in chapter

%%%%%%%%%%%%%%%%%%%%%%%%%%%%%%%%%%%%%%%%%%%%%%%%%%%%%%%%%%%%%%%%%%%%%%%%%%%
% Assignment 7.1: Chi-square testing by hand
%%%%%%%%%%%%%%%%%%%%%%%%%%%%%%%%%%%%%%%%%%%%%%%%%%%%%%%%%%%%%%%%%%%%%%%%%%%

\handassignment{Assignment 7.1: Chi-square testing by hand}

A small bed and breakfast wants to know if the \concept{distribution} of the number of guests is changing. They compare their historical \concept{distribution} with the number of guests in 2019 and want to show, with 95\% confidence, that the shape of the  2019 \concept{distribution} has changed with respect to the past years. \\

\question{
    Formulate the \concept{null hypothesis} $H_0$ and the \concept{alternative hypothesis} $H_1$ for this test in words.
}

\emptyanswerbox{
    $H_0$: \rule{.9\textwidth}{0.4pt}
    \answerbreak
    $H_1$: \rule{.9\textwidth}{0.4pt}
}

The table below shows the historical \concept{distribution} of the bed and breakfast guests alongside the number of guests for 2019. 

\begin{center}
    \begin{tabular}{|r|c|c|c|c|c|}
    \cline{2-6} 
    \multicolumn{1}{c|}{} & Historical & Observed ($O$) & Expected ($E$) & $O - E$ & $\frac{(O - E)^2}{E}$ \tstrut\bstrut\\
    \hline
    Spring & 4.87 & 2.90 & & &  \tstrut\bstrut\\
    \hline
    Summer & 3.04 & 4.50 & & &  \tstrut\bstrut\\
    \hline
    Fall & 1.65 & 4.94 & & &  \tstrut\bstrut\\
    \hline
    Winter & 2.88 & 3.28 & & &  \tstrut\bstrut\\
    \noalign{\hrule height 2pt}
    Total & 2.31 & 4.73 & & &  \tstrut\bstrut\\
    \cline{1-4} \cline{6-6}
    \end{tabular}
\end{center}
\vspace*{0.5cm}

\question{
    Fill in the expected ($E$) column in the table and show how you calculated it below.
}

\twolineanswerbox

\clearpage % Page break

\question{
    Are you allowed to do a \concept{chi-square test} using these data? Explain why.
}

\hint{Take into account the assumptions of a \concept{chi-square test}.}

\twolineanswerbox

\question{
    Fill in the rest of the table to calculate the \concept{chi-square value} ($X^2$) for this sample.
}

\emptyanswerbox{
    $X^2$: \shortanswerline
}

The critical \concept{chi-square value} using a confidence of 95\% and 3 \concept{degrees of freedom} is $X^2_{0.05}[df = 3] = 7.815$. \\

\question{
    What is your conclusion on the basis of these results? Include the following elements:
            \begin{itemize}
        \item[$\square$] Show how the calculated \concept{Chi-square value} relates to the \concept{critical Chi-square value}.
    \item[$\square$] Discuss whether $H_0$ is rejected or not.
    \item[$\square$] Describe what this tells us about the historical \concept{distribution} and the 2019 \concept{distribution}.
    \item[$\square$] Describe what type of error is relevant \textit{(type-I or type-II)}.
\end{itemize}
}

\sixlineanswerbox

\clearpage % Page break

\question{
    Taking into account the confidence of our analysis, explain in what range the \concept{p-value} of this test must be.
}

\twolineanswerbox

\begin{minipage}{0.8\textwidth}
The bed and breakfast owner does not believe in your calculations and wants you to recalculate it in \texttt{R}. \\

Run the following code in \texttt{R} to import the data: \\
\end{minipage}%
\hfill%
\begin{minipage}{0.1\textwidth}
\includegraphics[width=\linewidth]{Files/Images/displaycode.pdf}
\end{minipage}
\vspace*{.1cm}

\codeblock{Observed   <- c(37, 27, 12, 8)\\
Historical <- c(0.3, 0.4, 0.15, 0.15)\\
Expected   <- c(25.2, 33.6, 12.6, 12.6)}

\question{
    Calculate the \concept{Chi-square value} for these data using \texttt{R}.
}

\rcodeanswertiny

\emptyanswerbox{
    $X^2$: \shortanswerline
}

Run the following code in \texttt{R} to find out the \concept{critical Chi-square value}: \\

\codeblock{qchisq(p = 0.95, df = 3)}

\question{
    Does your conclusion hold up when you recalculate the \concept{Chi-square value} in \texttt{R}?
}

\emptyanswerbox{
    \vspace*{-15pt}
    \begin{center}
        YES / NO
    \end{center}
}

Use the following code in \texttt{R} to perform a \concept{Chi-square test}: \\

\codeblock{{\color{dataset}\# chi-square test: x = observations p = model distribution}\\ 
{\color{dataset}\# rescale makes sure the model distribution adds up to 100\%}\\
chisq.test(x = Observed, p = Historical, rescale.p = TRUE)}

\clearpage % Page break

\question{
    Is the result what you expected?
}

\twolineanswerbox

As it turns out the function \rcode{chisq.test()} will always calculate the \concept{expected values} by itself. You therefore do not need to make sure your total amounts sum to one. Therefore, in the code after \rcode{p =} you can supply a distribution that adds up to one (and set \rcode{rescale.p = FALSE}), or expected amounts that \texttt{R} will recalculate into a distribution anyway. If you do not supply the \rcode{p} argument, \texttt{R} will test against the \concept{uniform distribution}. \\

Run the following code in \texttt{R} to store the results of the \concept{Chi-square test} in an object called \rcode{chisq}: \\

\codeblock{chisq <- chisq.test(x = Observed, p = Historical)}

\question{
    Find out how to extract the \concept{expected values} from the \rcode{chisq} object. Do they match the \concept{expected values} you wrote down in the table above?
}

\rcodeanswertiny

\twolineanswerbox

\clearpage % Page break